\section{Einführung}
\subsection{Überblick}
Dieses Dokument stellt den Systementwurf des EnergyChallenge-Systems dar. Im Folgenden werden diese Themen behandelt:
\begin{itemize}
\item Kurze Einleitung in die verwendeten Technologien
\item Beschreibung der beteiligten Komponenten und deren Verteilung auf Hardware
\item Strukturierung des Systems in einzelne Pakete
\item Aufbau und Beschreibung der in den Paketen enthaltenen Klassen
\item Beschreibung des Verhaltens von einzelnen Anwendungsf\"allen mit Hilfe von Sequenzdiagrammen
\end{itemize}
\subsection{Technologien}
Das von uns entworfene und hier beschriebene System verwendet die folgenden Technologien:
\paragraph{Android}
Android ist ein Betriebssystem für mobile Ger\"aten, wie Smartphones oder Tablets. Android Anwendungen, kurz Apps, sind in Java geschrieben und laufen in der Android Runtime, einer Android-eigenen Laufzeitumgebung (ab Android 5.0). Android stellt eine Vielzahl von Bibliotheken bereit, deren Verwendung das Entwickeln von Apps vereinfacht und teilweise auch notwendig ist.
\begin{itemize}
\item http://www.android.com
\end{itemize}

\paragraph{Java}
Java ist eine objektorientierte Programmiersprache, in der dieses Projekt größtenteils geschrieben ist. Dabei wird Java größtenteils durch Android-Bibliothken und Grooy-Code ergänzt.
\begin{itemize}
\item http://java.oracle.com
\end{itemize}

\paragraph{GRAILS}
GRAILS ist ein auf der Programmiersprache Groovy aufbauendes Web Application Framework, welches mit relativ geringem Aufwand die Möglichkeit bietet, Websites mit Datenbank Anbindung zu erstellen. Mit GRAILS werden auch sogenannte Groovy Server Pages (GSP) erzeugt, welche Inhalte in sonst statische Websites einbinden können.
\begin{itemize}
\item http://grails.org
\end{itemize}
\paragraph{GORM}
GRAILS Object-Relational Mapping (kurz GORM) ist GRAILS' integriertes ORM-Tool welches Hibernate 4 zum dauerhaften Speicheern von Daten in einer Datenbank verwendet.
\begin{itemize}
\item http://grails.org/doc/2.3.x/guide/GORM.html
\item http://www.hibernate.org
\end{itemize}

\paragraph{Apache Shiro}
Apache Shiro ist ein Java Security Framework, welches zur Autorisierung und Authentifizierung von Benutzern auf Websiten verwendet wird. Es wird das Apache Shiro Plugin für Grails verwendet, welches viele Vorgänge automatisiert und vereinfacht.
\begin{itemize}
\item http://shiro.apache.org
\item http://grails.org/plugin/shiro
\end{itemize}
