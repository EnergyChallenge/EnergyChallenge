\section{Klassen}
\subsection{Webseite}

\subsection{Android App}
Die Klassen in \emph{Activities}, \emph{Adapters} sowie die Klasse \emph{AccessServerTask} erweitern vordefinierte Android-Klassen. Hierdurch werden viele Methoden definiert, die die vordefinierten Methoden überschreiben, wie zum Beispiel \emph{onCreate} oder \emph{onDestroyView}. Diese werden im Klassendiagramm und der folgenden Beschreibung nicht explizit aufgeführt.
\subsubsection{Activities}
\paragraph{MainActivity} Die \emph{Activity}, die für die Darstellung der Hauptfunktionen zuständig ist. Das bedeutet in ihr wird die Navigation aufgerufen und die meisten Funktionen, die als \emph{Fragment} dargestellt werden.
\paragraph{LoginActivity} Die \emph{Activity} über die, die Anmeldung erfolgt.
\paragraph{SearchActivity} Dient zur Ausführung einer Suche und der Anzeige der Ergebnisse.
\paragraph{ProposalActivity} Dient zur Anzeige eines Energiesparvorschlags inklusive seiner Bewertungen und Kommentare. Ermöglicht außerdem das Bewerten und Kommentieren von Energiesparvorschlags.
\paragraph{TeamProfilActivity} Zeigt das Profil eines Benutzers an.
\paragraph{UserProfilActivity} Zeigt das Profil eines Teams an.
\paragraph{MainFragment} Darstellung des Haupt-\emph{Fragments}.
\paragraph{MyProfilFragment} Zeigt das eigene Profil an.
\paragraph{RankinglistFragment} Zeigt das \emph{Fragment} für die Benutzer- bzw. Teamrangliste an.
\paragraph{TeamRankingsTabFragment} Zeigt die aktuelle Team-Rangliste an.
\paragraph{UserRankingsTabFragment} Zeigt die aktuelle Benutzer-Rangliste an.
\paragraph{ProposalFragment} Zeigt alle Energiesparvorschläge an.
\paragraph{NavigationFragment} Die Navigation innerhalb der Haupt-\emph{Activity}.

\subsubsection{Adapters}
\paragraph{FavoriteActivitiesAdapter} Zuständig für die Anzeige der favorisierten Aktivitäten. Von hier aus sollen auch Aktivitäten ausgeführt werden können, wenn sie angeklickt werden.
\paragraph{ActivitiesAdapter} Zuständig für die Anzeige aller Aktivitäten. Bei einem Klick auf eine Aktivität wird diese ausgeführt.
\paragraph{UserRankingsAdapter} Zuständig für die Anzeige der Benutzer-Rangliste. Bei einem Klick auf einen Benutzer soll eine \emph{UserProfilActivity} erstellt werden, in der der angeklickte Benutzer ausgeben wird.
\paragraph{TeamRankingsAdapter} Zuständig für die Anzeige der Team-Rangliste. Bei einem Klick auf ein Team soll eine \emph{TeamProfilActivity} erstellt werden, in der das angeklickte Team ausgeben wird.
\paragraph{ProposalsAdapter} Zuständig für die Anzeige der Energiesparvorschläge. Bei einem Klick auf einen Vorschlag soll eine \emph{ProposalActivity} erstellt werden, in der das angeklickte Team ausgeben wird.

\subsubsection{Tasks}
\paragraph{AccessServerTask} Eine Klasse, die eine abstrakter Serveranfrage darstellt. Alle Serveranfragen-Klassen erben von dieser Klasse und implementieren die abstrakten Methoden \emph{createServerRequest()} und \emph{handleServerResponse()} in welchen eine Server-Anfrage gestellt wird und eine Server-Rückgabe verarbeitet wird. Die Klasse kümmert sich um den Verbindungsaufbau, die Antwort und das Parsen der Daten, so dass die abgeleiteten Methoden sich nicht mehr darum kümmern müssen.
\paragraph{GetTeamProfileTask} Die Klasse, die dafür zuständig ist, ein Teamprofil vom Server zu laden und in der \emph{TeamProfilActivity} anzuzeigen.
\paragraph{GetUserProfileTask} Die Klasse, die dafür zuständig ist, ein Benutzerprofil vom Server zu laden und in der \emph{UserProfilActivity} bzw. im \emph{MyProfilFragment} anzuzeigen.
\paragraph{GetTeamRakingTask} Die Klasse, die dafür zuständig ist, das Team-Ranking vom Server zu holen und an den \emph{TeamRankingsAdapter} weiterzureichen.
\paragraph{GetUserRakingTask} Die Klasse, die dafür zuständig ist, das Benutzer-Ranking vom Server zu holen und an den \emph{TeamRankingsAdapter} weiterzureichen.
\paragraph{GetActivitiesTask} Die Klasse, die dafür zuständig ist, die Liste aller Aktivitäten vom Server zu holen und an den \emph{ActivitiesAdapter} weiterzureichen.
\paragraph{GetFavoriteActivitiesTask} Die Klasse, die dafür zuständig ist, die Liste der favorisierten Aktivitäten vom Server zu holen und im \emph{MainFragment} anzuzeigen.
\paragraph{GetProposalsTask} Die Klasse, die dafür zuständig ist, die Energiesparvorschläge vom Server zu holen und an den \emph{ProposalsAdapter} weiterzureichen.
\paragraph{GetProposalTask} Die Klasse, die dafür zuständig ist, einen konkreten Energiesparvorschlag vom Server zu laden und in der \emph{ProposalActivity} anzuzeigen.
\paragraph{SearchTask} Die Klasse, die dafür zuständig ist, eine Suchanfrage an den Server zu stellen und das Ergebnis in der \emph{SearchActivity} anzuzeigen.
\paragraph{DoActivityTask} Die Klasse, die dafür zuständig ist, dem Server mitzuteilen, dass ein Benutzer eine Aktivität ausgeführt hat.
\paragraph{CommentProposalTask} Die Klasse, die dafür zuständig ist, dem Server mitzuteilen, dass ein Benutzer einen Energiesparvorschlag bewertet oder kommentiert hat.

\subsubsection{Utils}
\paragraph{IoX} Klasse, die Methoden für generelles Input/Output bereitstellt. Das ist zum aktuellen Zeitpunkt nur die Methode \emph{readInputStream}, die einen \emph{InputStream} in einen String umwandelt.
\paragraph{ServerRequest} Klasse, die eine Serveranfrage modelliert. Eine Serveranfrage besteht aus einem Adressaten auf dem Server und aus der eigentlichen Anfrage. Die eigentliche Anfrage ist ein JSON-Objekt.