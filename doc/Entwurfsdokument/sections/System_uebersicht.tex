\section{System \"Ubersicht}

\subsection{Komponetendiagramm}

\paragraph{Server} Diese Komponente ist das Hauptprogramm, das auf dem physikalischen Server läuft. Hier findet die Datenhaltung, -bereitstellung und die Generierung die Website statt.

\paragraph{Model} Die Modellierung sämtlicher Daten. Dazu gehört das Speichern und der Zugriff auf diese.

\paragraph{Profil Management} Die Modellierung der Profildaten, also der Daten für Benutzer und Teams. Dazu gehört das Speichern und der Zugriff auf diese.

\paragraph{Benutzer Management} Die Modellierung der Benutzerdaten. Dazu gehört das Speichern und der Zugriff auf diese.

\paragraph{Team Management} Die Modellierung der Teamdaten. Dazu gehört das Speichern und der Zugriff auf diese.

\paragraph{Aktivität Management} Die Modellierung der Daten, der Aktivitäten. Dazu gehört das Speichern und der Zugriff auf diese.

\paragraph{Vorschläge Management}  Die Modellierung der Energiesparvorschläge, der Aktivitäten. Dazu gehört das Speichern und der Zugriff auf diese.

\paragraph{Statistiken} Die Modellierung der Statistiken, der Aktivitäten. Dazu gehört das Speichern und der Zugriff auf diese.

\paragraph{Website} Präsentiert die Daten des Models erlaubt Interaktionen.

\paragraph{Controller (Website)} Reagiert auf Benutzereingaben beziehungsweise -anfragen und kümmert sich darum, die benötigten Daten vom Model zu erhalten.

\paragraph{View (Website)} Präsentiert die Daten des Models.

\paragraph{App Controller} Zuständig für die Interaktion der App mit dem Server. Auf eine Anfrage der App werden Daten aus dem Model mithilfe von JSON für die App bereitgestellt.

\paragraph{Android App} Die native Android App, die in erster Linie zur Präsentation von Daten, die auf dem Server liegen, dient.

\paragraph{View (Android App)} Präsentiert die Daten innerhalb der App.

\paragraph{Controller (Android App)} Reagiert auf Benutzereingaben beziehungsweise -anfragen und kümmert sich darum die geforderte Aktion an den Server oder das App-Model weiterzuleiten.

\paragraph{Model (Android App)} Kümmert sich um die Datenhaltung innerhalb der App. Das ist zuerst nur die Information darüber, welcher Benutzer angemeldet ist, kann jedoch später noch um Cacheing-Funktionen erweitert werden, um nicht immer alle Daten vom Server holen zu müssen.

\subsection{Verteilungsdiagramm}

\subsubsection{Android-Gerät}\\
Das Android-Gerät ist das Arbeitswerkzeug für die Benutzer und die Administratoren. Auf dem Gerät ist eine native Android-Applikation installiert, mit der Daten eingegeben und abgerufen werden können. Der Benutzer muss bereits registriert sein, um die Applikation nutzen zu können.\\
\subsubsection{Computer}\\
Benutzer können mit Hilfe eines Computers über einen Webbrowser auf eine Webseite zugreifen, welche, nach erfolgreichem einloggen, verschiedene Funktionen zur Aktivitäts- und Profilverwaltung anbietet.Des weiteren können hier auch anonymisierte Statistiken angezeigt und diese statistischen Daten exportiert werden.\\
\subsubsection{Server}\\
Der Server stellt die eben erwähnte Webseite bereit. Hier werden gegebenenfalls Datenanfragen von den Android-Geräten verarbeitet.\\
\subsection{Paketdiagramm}