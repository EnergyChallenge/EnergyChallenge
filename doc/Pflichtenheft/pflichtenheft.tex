\documentclass[10pt,a4paper]{article}
\usepackage[latin1]{inputenc}
\usepackage{amsmath}
\usepackage{amsfonts}
\usepackage{amssymb}
\usepackage{graphicx}


\author{Grupe 02}
\title{PflichtenHeft}
\begin{document}
% Title Page
\maketitle
% Inhaltsverzeichniss
\tableofcontents
%------------------------------------------------
% Zielbestimmung
%------------------------------------------------
\section{Zielbestimmung}
	\subsection{Musskriterien}
	\begin{itemize}
		\item Webanwendung
		\begin{itemize}
			\item Startseite
			\item am Corporate Design der Universitaet und Klik-Webseite orientiert
			\item ohne Anmeldung ist nur die Startseite und die Aktivitaetsliste einsehbar
			\item Registrierung fuer Teilnehmer mit Benutzername, Email-Adresse und Dienststelle/Studiengang erfordert.
			\item Jeder Teilnehmer und jedes Team hat ein oeffentlich einsehbares Profil (Punkte, Aktivitaeten, Foto,...) %TODO
			\item Teilnehmer sammeln Punkte, und koennen diese einsehn
			\item Teilnehmer koennen sich zu Teams zusammenschliessen um gemeinsam Punkte zusammeln, in diese kann eingeladen werden.
			\item Teilnehmer koennen nur in einem Team Mitglied sein
			\item Teilnehmer koennen Energiesparvorschlaege einreichen
			\item Teilnehmer koennen Aktivitaeten auswaehlen und erledigen
			\item Aktivitaeten haben einen Zeitramen, Heufigkeit, Kategorie und Punkte
			\item Aktivitaeten muessen nach Auswahl bestaetigt werden
			\item Es gibt Challenges die Zeitlich begrenzt sind
			\item An Challenges nimmt jeder automatisch teil
			
			\item Es gibt eine Uebersich ueber alle Energiesparvorschlaege
			\item Energiesparvorschlaege koennen kommentiert werden
			\item Energiesparvorschlaege koennen bewertet werden
			\item Es gibt eine Uebersicht ueber alle Teilnehmer und alle Teams (sortiert nach Name / Punkten) %TODO
			\item Grafische annonymisierte Statistiken:
			\begin{itemize}
				\item Besucherzahl
				\item gesammelte Punkte
				\item erledigte Aktivitaeten
				\item beliebteste Aktivitaeten
			\end{itemize}
		\end{itemize}
		% ----------------------------------------------------------------
		\item Android-App
		\begin{itemize}
			\item Einmaliges login nach registrierung auf der Webseite
			\item Profilseiten sind einsehbar
			\item Ranking von Teilnehmern und Teams einsehbar
			\item Uebersicht aller Aktivitaeten
			\item Aktivitaeten auswaehlen und erledigen
			\item Liste der ausgewaehlten Aktiviaeten
			\item Suchfunktion fuer Teilnehmer, Teams und Aktivitaeten
			
		\end{itemize}
		% ----------------------------------------------------------------
		\item Admin Bereich
		\begin{itemize}
			\item Challenges erstellen
			\item Aktivitaeten erstellen und bearbeiten
			\item Energiesparvorschlaege annehmen
			\item Emails an alle Teilnehmer
			\item Teilnehmer und Teams sperren.
		\end{itemize}
	\end{itemize}
	\subsection{Sollkriterien}
	\begin{itemize}
			\item Profilbild fuer Teilnehmer und Gruppen
			\item Individueller Gruppen Vorschlag nach Registrierung
			\item Admin kann Teilnehmer und Teams reaktivieren und loeschen
			\item Admin kann Teams und Teilnehmer verwarnen
			\item Challenges in App implementiert
			\item Einstellbare Benachrichtigungen fuer App, Browser und Emails
		\end{itemize}
	\subsection{Kannkriterien}
	\begin{itemize}
			\item Profile lassen personalisieren
			\item Admin kann Emails an Einzelteilnehmer und Gruppen senden.
			\item Benutzer wechlsen bei der App
	\end{itemize}
	\subsection{Differentierungskriterien}
	\begin{itemize}
		\item keine direkte Kommunikation unter den Teilnehmern
		\item keine Popups in App und Browser, statt dessen Benachrichitungen 
	\end{itemize}

%------------------------------------------------
% Produkteinsatz
%------------------------------------------------
\section{Produkteinsatz}
\subsection{Answendungsbereich}
\subsection{Zielgruppe}
\subsection{Betriebsbedingungen}

%------------------------------------------------
% Produktumgebung
%------------------------------------------------
\section{Produktumgebung}
\subsection{Software}
\subsection{Hardware}
\subsection{Orgware}

%------------------------------------------------
% Produktuebersicht
%------------------------------------------------
\section{Produkt\"ubersicht}

%------------------------------------------------
% Akteure
%------------------------------------------------
\section{Akteure}

%------------------------------------------------
% Produktfunktion
%------------------------------------------------
\section{Produktfunktion}
\subsection{Webseite}

	\subsubsection{Startseite}
	\begin{figure}[h]
		\includegraphics[width=\linewidth,]{gfx/webseite/startseite.pdf}
	\end{figure}
	\begin{tabular}{|l|p{.5\linewidth}|}
	\hline Use Case Nummer & 1.1 \\ 
	\hline Use Case Name & Startseite \\ 
	\hline Initiirender Akteur & Benutzer \\
	\hline Weitere Akteure &  \\
	\hline Kurzbeschreibung & der Benutzer ruft die Startseite im Browser auf und hat die M\"oglichkeit sich zu registrieren, einzuloggen und das Passwort neu anzufordern \\
	\hline Vorbedingung & nicht eingeloggt \\
	\hline Nachbedingung &  \\
	\hline \multicolumn{2}{|c|}{Funktionalitaet des UseCases}\\
	\hline Ablauf & Bentuzer ruft die Webseite auf \\
	\hline Alternativen &  \\
	\hline Ausnahmen &  \\
	\hline Benuzte Use Cases &  \\
	\hline \multicolumn{2}{|c|}{Weitere Inforamtionen} \\
	\hline Spezielle Anforderungen &  \\
	\hline Annahmen &  \\
	\hline
	\end{tabular} 
	\subsubsection{Login}
		\begin{tabular}{|l|p{.5\linewidth}|}
		\hline Use Case Nummer & 1.1.1 \\ 
		\hline Use Case Name & Login \\ 
		\hline Initiirender Akteur & Benutzer \\
		\hline Weitere Akteure & Admin \\
		\hline Kurzbeschreibung & Mit dem Login stehen die eigentlichen Funktionen im Browser zur Verf\"ugung \\
		\hline Vorbedingung & Benutzer ist registriert und nicht nicht eingeloggt \\
		\hline Nachbedingung & Benutzer ist eingeloggt \\
		\hline \multicolumn{2}{|c|}{Funktionalitaet des UseCases}\\
		\hline Ablauf & \begin{itemize}
			\item Der Benutzer gibt Email-Adresse und Passwort ein
			\item Der Benutzer klickt den Login Button
		\end{itemize} \\
		\hline Alternativen &  \\
		\hline Ausnahmen & \begin{itemize}
			\item Das Passwort ist falsch, es wird eine Fehlermeldung angezeigt und die Option Passwort vergessen angeboten
			\item Die Email-adresse ist noch nicht registriert, es wird eine Fehlermedldung angezeigt, und die Registrierung angeboten
		\end{itemize} \\
		\hline Benuzte Use Cases &  \\
		\hline \multicolumn{2}{|c|}{Weitere Inforamtionen} \\
		\hline Spezielle Anforderungen &  \\
		\hline Annahmen &  \\
		\hline
		\end{tabular}
			 
	\subsubsection{Registrieren}
		\begin{tabular}{|l|p{.5\linewidth}|}
		\hline Use Case Nummer & 1.1.2 \\ 
		\hline Use Case Name & Registrieren \\ 
		\hline Initiirender Akteur & Benutzer \\
		\hline Weitere Akteure &  \\
		\hline Kurzbeschreibung & Ein neuer Benutzer kann sich mit seiner Email-Adresse anmelden \\
		\hline Vorbedingung & Benutzer hat auf Registrieren geklickt \\
		\hline Nachbedingung & Benutzer ist eingeloggt \\
		\hline \multicolumn{2}{|c|}{Funktionalitaet des UseCases}\\
		\hline Ablauf & \begin{itemize}
			\item Benutzer gibt Email-Adresse und Passwort ein
			\item Benutzer w\"ahlt fakult\"at
			\item Benutzer klickt "jetzt mitmachen"
		\end{itemize} \\
		\hline Alternativen &  \\
		\hline Ausnahmen & \begin{itemize}
			\item Email-adresse ist schon vergeben, es wird eine Fehlermeldung angezeigt
			\item das Passwort ist zu kurz, es wird eine Fehlermeldung angezeit
			\item Keine Fakult\"at gewahlt, es wir eine Fehlermeldung angezeigt
		\end{itemize} \\
		\hline Benuzte Use Cases &  \\
		\hline \multicolumn{2}{|c|}{Weitere Inforamtionen} \\
		\hline Spezielle Anforderungen &  \\
		\hline Annahmen &  \\
		\hline
		\end{tabular} 
		
	\subsubsection{Passwort vergessen}
		\begin{tabular}{|l|p{.5\linewidth}|}
		\hline Use Case Nummer & 1.1.3 \\ 
		\hline Use Case Name & Passwort vergessen \\ 
		\hline Initiirender Akteur & Benutzer \\
		\hline Weitere Akteure &  \\
		\hline Kurzbeschreibung & Ein Benutzer hat die M\"oglichkeit sein Passwort neu anzufordern \\
		\hline Vorbedingung & Benutzer ist nicht eingeloggt \\
		\hline Nachbedingung & Benutzer hat eine Email bekommen \\
		\hline \multicolumn{2}{|c|}{Funktionalitaet des UseCases}\\
		\hline Ablauf & \begin{itemize}
			\item Benutzer gibt seine Email-Adresse an
			\item Benutzer klickt auf "jetzt Passwort anfordern"
		\end{itemize} \\
		\hline Alternativen &  \\
		\hline Ausnahmen & Email-Adresse ist nicht registriert, es wird eine Fehlermeldung angezeigt \\
		\hline Benuzte Use Cases &  \\
		\hline \multicolumn{2}{|c|}{Weitere Inforamtionen} \\
		\hline Spezielle Anforderungen &  \\
		\hline Annahmen &  \\
		\hline
		\end{tabular}
	\subsubsection{Fehlermeldung}
		\begin{tabular}{|l|p{.5\linewidth}|}
		\hline Use Case Nummer & 1.1.4 \\ 
		\hline Use Case Name & Fehlermeldung \\ 
		\hline Initiirender Akteur & Benutzer \\
		\hline Weitere Akteure &  \\
		\hline Kurzbeschreibung & Es wird dem Benutzer eine Fehlermeldung angezeigt und beleibt auf der aktuellen Seite \\
		\hline Vorbedingung &  \\
		\hline Nachbedingung & Fehlermeldung wir angezeigt \\
		\hline \multicolumn{2}{|c|}{Funktionalitaet des UseCases}\\
		\hline Ablauf & Fehlermeldung wird in die Aktuelle Seite eingebettet \\
		\hline Alternativen &  \\
		\hline Ausnahmen &  \\
		\hline Benuzte Use Cases &  \\
		\hline \multicolumn{2}{|c|}{Weitere Inforamtionen} \\
		\hline Spezielle Anforderungen &  \\
		\hline Annahmen &  \\
		\hline
		\end{tabular} 
<<<<<<< HEAD
	 

=======
%------------------
\subsection{Landingpage}
%-----------------
\subsection{Aktivit\"aten}
>>>>>>> a66e7062aa58d82925d423a0aea5a264a3775ae3
	\subsubsection{Aktivit\"atenseite}
	\begin{tabular}{|l|p{.5\linewidth}|}
	\hline Use Case Nummer & 1.3 \\ 
	\hline Use Case Name & Aktivit\"atenseite \\ 
	\hline Initiierender Akteur & Benutzer \\
	\hline Weitere Akteure & Admin \\
	\hline Kurzbeschreibung & Der Benutzer kann eine Liste von verf\"ugbaren, ausgew\"ahlten und erledigten Aktivit\"aten ansehen, sowie Aktivit\"aten ausw\"ahlen, abw\"ahlen oder als erledigt markieren \\
	\hline Vorbedingung & Die Aktivit\"atenseite ist im Browser aufgerufen \\
	\hline Nachbedingung & Die Aktivit\"atenseite ist im Browser aufgerufen \\
	\hline \multicolumn{2}{|c|}{Funktionalitaet des UseCases}\\
	\hline Ablauf & 1. Benutzer ruft die Aktivit\"atenseite auf \\
	\hline Alternativen & - \\
	\hline Ausnahmen & Die Aktivit\"atenseite ist nicht verf\"ugbar \\
	\hline Benutzte Use Cases & - \\
	\hline \multicolumn{2}{|c|}{Weitere Inforamtionen} \\
	\hline Spezielle Anforderungen & - \\
	\hline Annahmen & - \\
	\hline
	\end{tabular} 
	\subsubsection{Aktivit\"aten ansehen}
	\begin{tabular}{|l|p{.5\linewidth}|}
	\hline Use Case Nummer & 1.3.1 \\ 
	\hline Use Case Name & Aktivit\"aten ansehen \\ 
	\hline Initiierender Akteur & Benutzer \\
	\hline Weitere Akteure & Admin \\
	\hline Kurzbeschreibung & Der Benutzer kann die auf der Seite dargestellte Liste von verf\"ugbaren Aktivit\"aten einsehen \\
	\hline Vorbedingung & Die Aktivit\"atenseite ist im Browser aufgerufen \\
	\hline Nachbedingung & Die Aktivit\"atenseite ist im Browser aufgerufen \\
	\hline \multicolumn{2}{|c|}{Funktionalitaet des UseCases}\\
	\hline Ablauf & 1. Der Benutzer sieht die Liste der Aktivit\"aten an \\
	\hline Alternativen & - \\
	\hline Ausnahmen & Die Aktivit\"atenliste ist nicht verf\"ugbar \\
	\hline Benutzte Use Cases & - \\
	\hline \multicolumn{2}{|c|}{Weitere Inforamtionen} \\
	\hline Spezielle Anforderungen & - \\
	\hline Annahmen & - \\
	\hline
	\end{tabular} 
	\subsubsection{Aktivit\"aten ausw\"ahlen}
	\begin{tabular}{|l|p{.5\linewidth}|}
	\hline Use Case Nummer & 1.3.2 \\ 
	\hline Use Case Name & Aktivit\"aten ausw\"ahlen \\ 
	\hline Initiierender Akteur & Benutzer \\
	\hline Weitere Akteure & Admin \\
	\hline Kurzbeschreibung & Der Benutzer kann eine oder mehrere der verf\"ugbaren Aktivit\"aten ausw\"ahlen \\
	\hline Vorbedingung & Die Aktivit\"atenseite ist im Browser aufgerufen und die Liste von Aktivit\"aten wird angezeigt \\
	\hline Nachbedingung & Die Aktivit\"atenliste wird angezeigt, die ausgew\"ahlten Aktivit\"aten sind speziell hervorgehoben \\
	\hline \multicolumn{2}{|c|}{Funktionalitaet des UseCases}\\
	\hline Ablauf & \begin{itemize}
			\item 1. Benutzer w\"ahlt eine oder mehrere Aktivit\"aten aus
			\item 2. Die ausgew\"ahlten Aktivit\"aten werden als ausgew\"ahlt hervorgehoben
		\end{itemize} \\
	\hline Alternativen & - \\
	\hline Ausnahmen & Die Aktivit\"atenliste ist nicht verf\"ugbar \\
	\hline Benutzte Use Cases & 1.3.1 Aktivit\"aten ansehen \\
	\hline \multicolumn{2}{|c|}{Weitere Inforamtionen} \\
	\hline Spezielle Anforderungen & - \\
	\hline Annahmen & Der Benutzer w\"ahlt nur Aktivit\"aten aus, die noch nicht ausgew\"ahlt sind \\
	\hline
	\end{tabular}
	\subsubsection{Aktivit\"aten als erledigt abhaken}
	\begin{tabular}{|l|p{.5\linewidth}|}
	\hline Use Case Nummer & 1.3.2.1 \\ 
	\hline Use Case Name & Aktivit\"aten als erledigt abhaken \\ 
	\hline Initiierender Akteur & Benutzer \\
	\hline Weitere Akteure & Admin \\
	\hline Kurzbeschreibung & Der Benutzer kann ausgew\"ahlte Aktivit\"aten, die er durchgef\"uhrt hat, als erledigt abhaken \\
	\hline Vorbedingung & Der Benutzer hat bereits eine oder mehrere Aktivit\"aten ausgew\"ahlt \\
	\hline Nachbedingung & Die abgehakte/n Aktivit\"at/en werden als solche markiert und dem Benutzer die entsprechende Punktzahl gutgeschrieben \\
	\hline \multicolumn{2}{|c|}{Funktionalitaet des UseCases}\\
	\hline Ablauf & \begin{itemize}
			\item 1. Benutzer hakt eine oder mehrere der ausgew\"ahlten Aktivit\"aten als erledigt ab
			\item 2. Die abgehakten Aktivit\"aten werden als erledigt markiert und dem Benutzer die entsprechenden Punkte gutgeschrieben
		\end{itemize} \\
	\hline Alternativen & - \\
	\hline Ausnahmen & Die Aktivit\"at ist in dem Zeitraum bzw. derzeit nicht verf\"ugbar \\
	\hline Benutzte Use Cases & 1.3.1 Aktivit\"aten ansehen \\
	\hline \multicolumn{2}{|c|}{Weitere Informationen} \\
	\hline Spezielle Anforderungen & - \\
	\hline Annahmen & - \\
	\hline
	\end{tabular} 
	\subsubsection{Aktivit\"aten abw\"ahlen}
	\begin{tabular}{|l|p{.5\linewidth}|}
	\hline Use Case Nummer & 1.3.2.2 \\ 
	\hline Use Case Name & Aktivit\"aten abw\"ahlen \\ 
	\hline Initiierender Akteur & Benutzer \\
	\hline Weitere Akteure & Admin \\
	\hline Kurzbeschreibung & Der Benutzer kann eine oder mehrere Aktivit\"aten, die er zuvor ausgew\"ahlt hat, wieder abw\"ahlen \\
	\hline Vorbedingung & Es wurden bereits Aktivit\"aten ausgew\"ahlt und diese werden als solche angezeigt \\
	\hline Nachbedingung & Die abgew\"ahlten Aktivit\"aten sind nicht mehr als ausgew\"ahlt markiert \\
	\hline \multicolumn{2}{|c|}{Funktionalitaet des UseCases}\\
	\hline Ablauf & \begin{itemize}
			\item 1. Benutzer w\"ahlt eine oder mehrere der ausgew\"ahlten Aktivit\"aten ab
			\item 2. Die abgew\"ahlten Aktivit\"aten werden nicht mehr als ausgew\"ahlt hervorgehoben
		\end{itemize} \\
	\hline Alternativen & - \\
	\hline Ausnahmen & Die Aktivit\"atenliste ist nicht verf\"ugbar \\
	\hline Benutzte Use Cases & Aktivit\"aten ansehen \\
	\hline \multicolumn{2}{|c|}{Weitere Inforamtionen} \\
	\hline Spezielle Anforderungen & - \\
	\hline Annahmen & - \\
	\hline
	\end{tabular} 
	\subsubsection{Aktivit\"aten bearbeiten}
	\begin{tabular}{|l|p{.5\linewidth}|}
	\hline Use Case Nummer & 1.3.3 \\ 
	\hline Use Case Name & Aktivit\"aten bearbeiten \\ 
	\hline Initiierender Akteur & Admin \\
	\hline Weitere Akteure & - \\
	\hline Kurzbeschreibung & Der Admin kann einzelne Aktivit\"aten bearbeiten (Name, Punktzahl, Zeitraum, etc.) \\
	\hline Vorbedingung & Die Aktivit\"atenseite ist im Browser aufgerufen und eine Aktivit\"at zur Bearbeitung ausgew\"ahlt \\
	\hline Nachbedingung & Die bearbeitete Aktivit\"at wird in ihrer neuen Form in der Aktivit\"atenliste angezeigt \\
	\hline \multicolumn{2}{|c|}{Funktionalitaet des UseCases}\\
	\hline Ablauf & \begin{itemize}
			\item 1. Der Admin w\"ahlt eine Aktivit\"at zur Bearbeitung aus
			\item 2. Der Admin speichert die Aktivit\"at mit den vorgenommenen \"Anderungen
		\end{itemize} \\
	\hline Alternativen & Der Admin l\"oscht die Aktivit\"at \\
	\hline Ausnahmen & Die Aktivit\"atenliste ist nicht verf\"ugbar \\
	\hline Benutzte Use Cases & 1.3.1 Aktivit\"aten ansehen \\
	\hline \multicolumn{2}{|c|}{Weitere Inforamtionen} \\
	\hline Spezielle Anforderungen & - \\
	\hline Annahmen & - \\
	\hline
	\end{tabular} 
\subsection{Energiespavorschl\"age}
%---------------------------------------------
\subsection{Rangliste}
	\begin{figure}[h]
		\includegraphics[width=\linewidth,]{gfx/webseite/rangliste.pdf}
	\end{figure}
	\subsubsection{Rangliste Ansehen}
		\begin{tabular}{|l|p{.5\linewidth}|}
		\hline Use Case Nummer & 1.5 \\ 
		\hline Use Case Name & Rangliste Ansehen \\ 
		\hline Initiirender Akteur & Benutzer \\
		\hline Weitere Akteure &  \\
		\hline Kurzbeschreibung & Der Benutzer kann die Rangliste ansehen \\
		\hline Vorbedingung & Der Benutzer ist eingeloggt \\
		\hline Nachbedingung &  \\
		\hline \multicolumn{2}{|c|}{Funktionalitaet des UseCases}\\
		\hline Ablauf & Der Benutzer kann die Rangliste ansehen \\
		\hline Alternativen & Der Benutzer kann Suchen \\
		\hline Ausnahmen &  \\
		\hline Benuzte Use Cases &  \\
		\hline \multicolumn{2}{|c|}{Weitere Inforamtionen} \\
		\hline Spezielle Anforderungen &  \\
		\hline Annahmen &  \\
		\hline
		\end{tabular}
	\subsubsection{Profil ansehen}
		\begin{tabular}{|l|p{.5\linewidth}|}
		\hline Use Case Nummer & 1.5.1 \\ 
		\hline Use Case Name & Profil ansehen \\ 
		\hline Initiirender Akteur & Benutzer \\
		\hline Weitere Akteure &  \\
		\hline Kurzbeschreibung & Eine Ansicht eines Profils von einem Team oder einem Benutzer \\
		\hline Vorbedingung & Es wurde ein Profil angeklickt \\
		\hline Nachbedingung & Das Profil wird angezeigt \\
		\hline \multicolumn{2}{|c|}{Funktionalitaet des UseCases}\\
		\hline Ablauf & \begin{itemize}
			\item Benuzter klickt auf ein Profil
			\item Benutzer wird Profil angezeigt
		\end{itemize} \\
		\hline Alternativen &  \\
		\hline Ausnahmen &  \\
		\hline Benuzte Use Cases &  \\
		\hline \multicolumn{2}{|c|}{Weitere Inforamtionen} \\
		\hline Spezielle Anforderungen &  \\
		\hline Annahmen &  \\
		\hline
		\end{tabular}
	\subsubsection{Benutzer/Team Ansicht wechseln}
		\begin{tabular}{|l|p{.5\linewidth}|}
		\hline Use Case Nummer & 1.5.2 \\ 
		\hline Use Case Name & Benutzer/Team Ansicht wechseln \\ 
		\hline Initiirender Akteur & Benutzer \\
		\hline Weitere Akteure &  \\
		\hline Kurzbeschreibung & Der Benutzer kann zwischen der Benutzer Rangliste und der Team Rangliste hin und her schalten \\
		\hline Vorbedingung & Rangliste wird angezeigt \\
		\hline Nachbedingung & Ansicht wurde umgeschaltet \\
		\hline \multicolumn{2}{|c|}{Funktionalitaet des UseCases}\\
		\hline Ablauf & \begin{itemize}
			\item Benutzer klickt auf "Team"
			\item Team Ansicht wird angezeigt
		\end{itemize} \\
		\hline Alternativen & \begin{itemize}
					\item Benutzer klickt auf "Benutzer"
					\item Benutzer Ansicht wird angezeigt
				\end{itemize} \\
		\hline Ausnahmen &  \\
		\hline Benuzte Use Cases &  \\
		\hline \multicolumn{2}{|c|}{Weitere Inforamtionen} \\
		\hline Spezielle Anforderungen &  \\
		\hline Annahmen &  \\
		\hline
		\end{tabular}
		
		\subsubsection{Suche}
				\begin{tabular}{|l|p{.5\linewidth}|}
				\hline Use Case Nummer & 1.5.3 \\ 
				\hline Use Case Name & Suche \\ 
				\hline Initiirender Akteur & Benutzer \\
				\hline Weitere Akteure &  \\
				\hline Kurzbeschreibung & Der Benutzer kann nach Teams und Benutzern suchen \\
				\hline Vorbedingung &  \\
				\hline Nachbedingung & Suchergebnisse werden angezeigt \\
				\hline \multicolumn{2}{|c|}{Funktionalitaet des UseCases}\\
				\hline Ablauf & \begin{itemize}
					\item Benutzer gibt Suchanfrage ein
					\item Benutzer klickt suchen
					\item Suchergebnisse werden angezeigt
				\end{itemize} \\
				\hline Alternativen &  \\
				\hline Ausnahmen & Wenn keine Ergebnisse gefunden werden wird dies angezeigt \\
				\hline Benuzte Use Cases &  \\
				\hline \multicolumn{2}{|c|}{Weitere Inforamtionen} \\
				\hline Spezielle Anforderungen &  \\
				\hline Annahmen &  \\
				\hline
				\end{tabular}
%-----------------------------
\subsection{Statistiken}
%-----------------------------
\subsection{AdminBereich}
%-----------------------------
\subsection{Profilseite}
	\subsubsection{Profilseite \"andern}
		\begin{tabular}{|l|p{.5\linewidth}|}
		\hline Use Case Nummer & 1.8 \\ 
		\hline Use Case Name & Profilseite \"andern \\ 
		\hline Initiirender Akteur & Benutzer \\
		\hline Weitere Akteure & Admin \\
		\hline Kurzbeschreibung & Der Benutzer kann verschiedene Informationen seines Profils bearbeiten, wie Profilbild, Passwort, Benachrichtigungseinstellungen oder Fakult�t und besitzt die M�glichkeit sein Profil zu l�schen \\
		\hline Vorbedingung & Die Profilseite ist im Browser aufgerufen \\
		\hline Nachbedingung & Die Profilseite wird mit den vorgenommenen \"Anderungen angezeigt oder ist gel�scht worden \\
		\hline \multicolumn{2}{|c|}{Funktionalitaet des UseCases}\\
		\hline \hline Ablauf & \begin{itemize}
					\item 1. Der Benutzer w\"ahlt eine Information, die er bearbeiten m�chte aus und bearbeitet diese
					\item 2. Der Benutzer speichert die ge�nderten Informationen und diese werden ins System �bernommen
				\end{itemize}
		\hline Alternativen & - \\
		\hline Ausnahmen & - \\
		\hline Benuzte Use Cases & - \\
		\hline \multicolumn{2}{|c|}{Weitere Informationen} \\
		\hline Spezielle Anforderungen &  \\
		\hline Annahmen &  \\
		\hline
		\end{tabular}
		
			\subsubsection{Profil sperren}
		\begin{tabular}{|l|p{.5\linewidth}|}
		\hline Use Case Nummer & 1.8.1 \\ 
		\hline Use Case Name & Profil sperren \\ 
		\hline Initiirender Akteur & Admin \\
		\hline Weitere Akteure & - \\
		\hline Kurzbeschreibung & Der Admin kann ein Profil sperren, wenn irgendeine Form von Missverhalten vorliegt \\
		\hline Vorbedingung & Die Profilseite ist im Browser aufgerufen \\
		\hline Nachbedingung & Die Profilseite wird dem Admin als gesperrt angezeigt \\
		\hline \multicolumn{2}{|c|}{Funktionalitaet des UseCases}\\
		\hline \hline Ablauf & \begin{itemize}
					\item 1. Der Admin \"offnet die Profilseite des zu sperrenden Profils
					\item 2. Der Admin sperrt das entsprechende Profil, welches im Anschluss f�r Benutzer nicht mehr sichtbar und verwendbar ist
				\end{itemize}
		\hline Alternativen & - \\
		\hline Ausnahmen & - \\
		\hline Benuzte Use Cases & - \\
		\hline \multicolumn{2}{|c|}{Weitere Informationen} \\
		\hline Spezielle Anforderungen &  \\
		\hline Annahmen &  \\
		\hline
		\end{tabular}
		
			\subsubsection{Profil reaktivieren}
		\begin{tabular}{|l|p{.5\linewidth}|}
		\hline Use Case Nummer & 1.8.1.1 \\ 
		\hline Use Case Name & Profil reaktivieren \\ 
		\hline Initiierender Akteur & Admin \\
		\hline Weitere Akteure & - \\
		\hline Kurzbeschreibung & Admin kann ein zuvor gesperrtes Profil reaktivieren \\
		\hline Vorbedingung & Die gesperrte Profilseite ist im Browser aufgerufen \\
		\hline Nachbedingung & Die Profilseite ist wieder sichtbar \\
		\hline \multicolumn{2}{|c|}{Funktionalitaet des UseCases}\\
		\hline \hline Ablauf & \begin{itemize}
					\item 1. Der Admin ruft ein gesperrtes Profil auf
					\item 2. Der Admin reaktiviert das Profil, sodass es f�r alle Benutzer wieder sichtbar ist und der Profilinhaber wieder alle Funktionen nutzen kann
				\end{itemize}
		\hline Alternativen & - \\
		\hline Ausnahmen & - \\
		\hline Benuzte Use Cases & - \\
		\hline \multicolumn{2}{|c|}{Weitere Informationen} \\
		\hline Spezielle Anforderungen &  \\
		\hline Annahmen &  \\
		\hline
		\end{tabular}
		
			\subsubsection{Profilinhaber zum Admin bef\"ordern}
		\begin{tabular}{|l|p{.5\linewidth}|}
		\hline Use Case Nummer & 1.8.2 \\ 
		\hline Use Case Name & Profilinhaber zum Admin bef\"ordern \\ 
		\hline Initiirender Akteur & Admin \\
		\hline Weitere Akteure & Benutzer, der zum Admin wird \\
		\hline Kurzbeschreibung & Ein Admin kann einen Benutzer zum Admin ernennen und ihm so Administratorzugriff gew�hren \\
		\hline Vorbedingung & Die Profilseite ist im Browser aufgerufen \\
		\hline Nachbedingung & Die Profilseite wird im Browser angezeigt und der Benutzer ist im System als Admin registriert \\
		\hline \multicolumn{2}{|c|}{Funktionalitaet des UseCases}\\
		\hline \hline Ablauf & \begin{itemize}
					\item 1. Der Admin w�hlt das Profil des Benutzers aus, den er zum Admin bef�rdern m�chte
					\item 2. Der Admin bef�rdert den Benutzer �ber einen Men�punkt zum Administrator
				\end{itemize}
		\hline Alternativen & - \\
		\hline Ausnahmen & - \\
		\hline Benuzte Use Cases & - \\
		\hline \multicolumn{2}{|c|}{Weitere Informationen} \\
		\hline Spezielle Anforderungen &  \\
		\hline Annahmen & Das Profil des zu bef�rdernden Benutzers ist nicht gesperrt \\
		\hline
		\end{tabular}
		
			\subsubsection{Profil l\"oschen}
		\begin{tabular}{|l|p{.5\linewidth}|}
		\hline Use Case Nummer & 1.8.3 \\ 
		\hline Use Case Name & Profil l\"oschen \\ 
		\hline Initiirender Akteur & Admin \\
		\hline Weitere Akteure & - \\
		\hline Kurzbeschreibung & Der Admin kann ausgew\"ahlte Profile l\"oschen \\
		\hline Vorbedingung & Die Profilseite ist im Browser aufgerufen \\
		\hline Nachbedingung & Das Profil ist aus dem System gel\"oscht \\
		\hline \multicolumn{2}{|c|}{Funktionalitaet des UseCases}\\
		\hline \hline Ablauf & \begin{itemize}
					\item 1. Der Admin w\"ahlt das zu l�schende Profil aus
					\item 2. Der Admin l\"oscht das Profil
				\end{itemize}
		\hline Alternativen & - \\
		\hline Ausnahmen & - \\
		\hline Benuzte Use Cases & - \\
		\hline \multicolumn{2}{|c|}{Weitere Informationen} \\
		\hline Spezielle Anforderungen & - \\
		\hline Annahmen & - \\
		\hline
		\end{tabular}
%-----------------------------
\subsection{Andoid App}


%TODO Muster:
\subsubsection{title}
		\begin{tabular}{|l|p{.5\linewidth}|}
		\hline Use Case Nummer &  \\ 
		\hline Use Case Name &  \\ 
		\hline Initiirender Akteur &  \\
		\hline Weitere Akteure &  \\
		\hline Kurzbeschreibung &  \\
		\hline Vorbedingung &  \\
		\hline Nachbedingung &  \\
		\hline \multicolumn{2}{|c|}{Funktionalitaet des UseCases}\\
		\hline Ablauf &  \\
		\hline Alternativen &  \\
		\hline Ausnahmen &  \\
		\hline Benuzte Use Cases &  \\
		\hline \multicolumn{2}{|c|}{Weitere Inforamtionen} \\
		\hline Spezielle Anforderungen &  \\
		\hline Annahmen &  \\
		\hline
		\end{tabular}

%------------------------------------------------
% Produktdaten
%------------------------------------------------
\section{Produktdaten}

%------------------------------------------------
% Produktleistung
%------------------------------------------------
\section{Produktleistung}

%------------------------------------------------
% Benutzeroberflache
%------------------------------------------------
\section{Benutzeroberfl\"ache}
\subsection{Webseite}
\subsection{Android App}
%------------------------------------------------
% Qualitaetsanforderung
%------------------------------------------------
\section{Qualit\"atsanforderung}

%------------------------------------------------
% Entwicklungsumgebung
%------------------------------------------------
\section{Entwicklungsumgebung}
\subsection{Software}
\begin{itemize}
	\item 
	\item Java 7
	\item Visual Paradigm Standard Edition 12
\end{itemize}
\subsection{Hardware}
\subsection{Orgware}
%------------------------------------------------
% Erganzungen
%------------------------------------------------
\section{Erg\"anzungen}

%------------------------------------------------
% Glossar
%------------------------------------------------
\section{Glossar}

Backend ist der Admin Bereich der Webseite

\end{document}