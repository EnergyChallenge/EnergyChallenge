\documentclass[10pt,a4paper]{article}
\usepackage[latin1]{inputenc}
\usepackage{amsmath}
\usepackage{amsfonts}
\usepackage{amssymb}
\usepackage{graphicx}
\author{Grupe 02}
\title{PflichtenHeft}
\begin{document}
% Title Page
\maketitle
% Inhaltsverzeichniss
\tableofcontents
%------------------------------------------------
% Zielbestimmung
%------------------------------------------------
\section{Zielbestimmung}
	\subsection{Musskriterien}
	\begin{itemize}
		\item Webanwendung
		\begin{itemize}
			\item Startseite
			\item am Corporate Design der Universitaet und Klik-Webseite orientiert
			\item ohne Anmeldung ist nur die Startseite und die Aktivitaetsliste einsehbar
			\item Registrierung fuer Teilnehmer mit Benutzername, Email-Adresse und Dienststelle/Studiengang erfordert.
			\item Jeder Teilnehmer und jedes Team hat ein oeffentlich einsehbares Profil (Punkte, Aktivitaeten, Foto,...) %TODO
			\item Teilnehmer sammeln Punkte, und koennen diese einsehn
			\item Teilnehmer koennen sich zu Teams zusammenschliessen um gemeinsam Punkte zusammeln, in diese kann eingeladen werden.
			\item Teilnehmer koennen nur in einem Team Mitglied sein
			\item Teilnehmer koennen Energiesparvorschlaege einreichen
			\item Teilnehmer koennen Aktivitaeten auswaehlen und erledigen
			\item Aktivitaeten haben einen Zeitramen, Heufigkeit, Kategorie und Punkte
			\item Aktivitaeten muessen nach Auswahl bestaetigt werden
			\item Es gibt Challenges die Zeitlich begrenzt sind
			
			\item Es gibt eine Uebersich ueber alle Energiesparvorschlaege
			\item Energiesparvorschlaege koennen kommentiert werden
			\item Energiesparvorschlaege koennen bewertet werden
			\item Es gibt eine Uebersicht ueber alle Teilnehmer und alle Teams (sortiert nach Name / Punkten) %TODO
			\item Grafische annonymisierte Statistiken:
			\begin{itemize}
				\item Besucherzahl
				\item gesammelte Punkte
				\item erledigte Aktivitaeten
				\item beliebteste Aktivitaeten
			\end{itemize}
		\end{itemize}
		% ----------------------------------------------------------------
		\item Android-App
		\begin{itemize}
			\item Einmaliges login nach registrierung auf der Webseite
			\item Profilseiten sind einsehbar
			\item Ranking von Teilnehmern und Teams einsehbar
			\item Uebersicht aller Aktivitaeten
			\item Aktivitaeten auswaehlen und erledigen
			\item Liste der ausgewaehlten Aktiviaeten
			\item Suchfunktion fuer Teilnehmer, Teams und Aktivitaeten
			
		\end{itemize}
		% ----------------------------------------------------------------
		\item Admin Bereich
		\begin{itemize}
			\item Challenges erstellen
			\item Aktivitaeten erstellen und bearbeiten
			\item Energiesparvorschlaege annehmen
			\item Emails an alle Teilnehmer
			\item Teilnehmer und Teams sperren.
		\end{itemize}
	\end{itemize}
	\subsection{Sollkriterien}
	\begin{itemize}
			\item Profilbild fuer Teilnehmer und Gruppen
			\item Individueller Gruppen Vorschlag nach Registrierung
			\item Admin kann Teilnehmer und Teams reaktivieren und loeschen
			\item Admin kann Teams und Teilnehmer verwarnen
			\item Challenges in App implementiert
			\item Einstellbare Benachrichtigungen fuer App, Browser und Emails
		\end{itemize}
	\subsection{Kannkriterien}
	\begin{itemize}
			\item Profile lassen personalisieren
			\item Admin kann Emails an Einzelteilnehmer und Gruppen senden.
			\item Benutzer wechlsen bei der App
	\end{itemize}
	\subsection{Differentierungskriterien}
	\begin{itemize}
		\item keine direkte Kommunikation unter den Teilnehmern
		\item keine Popups in App und Browser, statt dessen Benachrichitungen 
	\end{itemize}
%------------------------------------------------
% Produkteinsatz
%------------------------------------------------
\section{Produkteinsatz}
\subsection{Answendungsbereich}
\subsection{Zielgruppe}
\subsection{Betriebsbedingungen}
%------------------------------------------------
% Produktumgebung
%------------------------------------------------
\section{Produktumgebung}
\subsection{Software}
\subsection{Hardware}
\subsection{Orgware}
%------------------------------------------------
% Produktuebersicht
%------------------------------------------------
\section{Produkt\"ubersicht}
%------------------------------------------------
% Akteure
%------------------------------------------------
\section{Akteure}
%------------------------------------------------
% Produktfunktion
%------------------------------------------------
\section{Produktfunktion}
%------------------------------------------------
% Produktdaten
%------------------------------------------------
\section{Produktdaten}
%------------------------------------------------
% Produktleistung
%------------------------------------------------
\section{Produktleistung}
%------------------------------------------------
% Benutzeroberflache
%------------------------------------------------
\section{Benutzeroberfl\"ache}
\subsection{Webseite}
\subsection{Android App}
%------------------------------------------------
% Qualitaetsanforderung
%------------------------------------------------
\section{Qualit\"atsanforderung}
%------------------------------------------------
% Entwicklungsumgebung
%------------------------------------------------
\section{Entwicklungsumgebung}
\subsection{Software}
\subsection{Hardware}
\subsection{Orgware}
%------------------------------------------------
% Erganzungen
%------------------------------------------------
\section{Erg\"anzungen}
%------------------------------------------------
% Glossar
%------------------------------------------------
\section{Glossar}

\end{document}