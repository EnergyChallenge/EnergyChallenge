\documentclass[10pt,a4paper]{article}
\usepackage[latin1]{inputenc}
\usepackage{amsmath}
\usepackage{amsfonts}
\usepackage{amssymb}
\usepackage{graphicx}


\author{Grupe 02}
\title{PflichtenHeft}
\begin{document}
% Title Page
\maketitle
% Inhaltsverzeichniss
\tableofcontents
%------------------------------------------------
% Zielbestimmung
%------------------------------------------------
\section{Zielbestimmung}
	\subsection{Musskriterien}
	\begin{itemize}
		\item Webanwendung
		\begin{itemize}
			\item Startseite
			\item am Corporate Design der Universitaet und Klik-Webseite orientiert
			\item ohne Anmeldung ist nur die Startseite und die Aktivitaetsliste einsehbar
			\item Registrierung fuer Teilnehmer mit Benutzername, Email-Adresse und Dienststelle/Studiengang erfordert.
			\item Jeder Teilnehmer und jedes Team hat ein oeffentlich einsehbares Profil (Punkte, Aktivitaeten, Foto,...) %TODO
			\item Teilnehmer sammeln Punkte, und koennen diese einsehn
			\item Teilnehmer koennen sich zu Teams zusammenschliessen um gemeinsam Punkte zusammeln, in diese kann eingeladen werden.
			\item Teilnehmer koennen nur in einem Team Mitglied sein
			\item Teilnehmer koennen Energiesparvorschlaege einreichen
			\item Teilnehmer koennen Aktivitaeten auswaehlen und erledigen
			\item Aktivitaeten haben einen Zeitramen, Heufigkeit, Kategorie und Punkte
			\item Aktivitaeten muessen nach Auswahl bestaetigt werden
			\item Es gibt Challenges die Zeitlich begrenzt sind
			\item An Challenges nimmt jeder automatisch teil
			
			\item Es gibt eine Uebersich ueber alle Energiesparvorschlaege
			\item Energiesparvorschlaege koennen kommentiert werden
			\item Energiesparvorschlaege koennen bewertet werden
			\item Es gibt eine Uebersicht ueber alle Teilnehmer und alle Teams (sortiert nach Name / Punkten) %TODO
			\item Grafische annonymisierte Statistiken:
			\begin{itemize}
				\item Besucherzahl
				\item gesammelte Punkte
				\item erledigte Aktivitaeten
				\item beliebteste Aktivitaeten
			\end{itemize}
		\end{itemize}
		% ----------------------------------------------------------------
		\item Android-App
		\begin{itemize}
			\item Einmaliges login nach registrierung auf der Webseite
			\item Profilseiten sind einsehbar
			\item Ranking von Teilnehmern und Teams einsehbar
			\item Uebersicht aller Aktivitaeten
			\item Aktivitaeten auswaehlen und erledigen
			\item Liste der ausgewaehlten Aktiviaeten
			\item Suchfunktion fuer Teilnehmer, Teams und Aktivitaeten
			
		\end{itemize}
		% ----------------------------------------------------------------
		\item Admin Bereich
		\begin{itemize}
			\item Challenges erstellen
			\item Aktivitaeten erstellen und bearbeiten
			\item Energiesparvorschlaege annehmen
			\item Emails an alle Teilnehmer
			\item Teilnehmer und Teams sperren.
		\end{itemize}
	\end{itemize}
	\subsection{Sollkriterien}
	\begin{itemize}
			\item Profilbild fuer Teilnehmer und Gruppen
			\item Individueller Gruppen Vorschlag nach Registrierung
			\item Admin kann Teilnehmer und Teams reaktivieren und loeschen
			\item Admin kann Teams und Teilnehmer verwarnen
			\item Challenges in App implementiert
			\item Einstellbare Benachrichtigungen fuer App, Browser und Emails
		\end{itemize}
	\subsection{Kannkriterien}
	\begin{itemize}
			\item Profile lassen personalisieren
			\item Admin kann Emails an Einzelteilnehmer und Gruppen senden.
			\item Benutzer wechlsen bei der App
	\end{itemize}
	\subsection{Differentierungskriterien}
	\begin{itemize}
		\item keine direkte Kommunikation unter den Teilnehmern
		\item keine Popups in App und Browser, statt dessen Benachrichitungen 
	\end{itemize}

%------------------------------------------------
% Produkteinsatz
%------------------------------------------------
\section{Produkteinsatz}
\subsection{Answendungsbereich}
\subsection{Zielgruppe}
\subsection{Betriebsbedingungen}

%------------------------------------------------
% Produktumgebung
%------------------------------------------------
\section{Produktumgebung}
\subsection{Software}
\subsection{Hardware}
\subsection{Orgware}

%------------------------------------------------
% Produktuebersicht
%------------------------------------------------
\section{Produkt\"ubersicht}

%------------------------------------------------
% Akteure
%------------------------------------------------
\section{Akteure}

%------------------------------------------------
% Produktfunktion
%------------------------------------------------
\section{Produktfunktion}
\subsection{Webseite}
\begin{figure}[h]
	\includegraphics[width=\linewidth,]{gfx/webseite/startseite.pdf}	
\end{figure}
	\subsubsection{Startseite}
	\begin{tabular}{|l|p{.5\linewidth}|}
	\hline Use Case Nummer & 1.1 \\ 
	\hline Use Case Name & Startseite \\ 
	\hline Initiirender Akteur & Benutzer \\
	\hline Weitere Akteure &  \\
	\hline Kurzbeschreibung & der Benutzer ruft die Startseite im Browser auf und hat die M\"oglichkeit sich zu registrieren, einzuloggen und das Passwort neu anzufordern \\
	\hline Vorbedingung & nicht eingeloggt \\
	\hline Nachbedingung &  \\
	\hline \multicolumn{2}{|c|}{Funktionalitaet des UseCases}\\
	\hline Ablauf & Bentuzer ruft die Webseite auf \\
	\hline Alternativen &  \\
	\hline Ausnahmen &  \\
	\hline Benuzte Use Cases &  \\
	\hline \multicolumn{2}{|c|}{Weitere Inforamtionen} \\
	\hline Spezielle Anforderungen &  \\
	\hline Annahmen &  \\
	\hline
	\end{tabular} 
	\subsubsection{Login}
		\begin{tabular}{|l|p{.5\linewidth}|}
		\hline Use Case Nummer & 1.1.1 \\ 
		\hline Use Case Name & Login \\ 
		\hline Initiirender Akteur & Benutzer \\
		\hline Weitere Akteure & Admin \\
		\hline Kurzbeschreibung & Mit dem Login stehen die eigentlichen Funktionen im Browser zur Verf\"ugung \\
		\hline Vorbedingung & Benutzer ist registriert und nicht nicht eingeloggt \\
		\hline Nachbedingung & Benutzer ist eingeloggt \\
		\hline \multicolumn{2}{|c|}{Funktionalitaet des UseCases}\\
		\hline Ablauf & \begin{itemize}
			\item Der Benutzer gibt Email-Adresse und Passwort ein
			\item Der Benutzer klickt den Login Button
		\end{itemize} \\
		\hline Alternativen &  \\
		\hline Ausnahmen & \begin{itemize}
			\item Das Passwort ist falsch, es wird eine Fehlermeldung angezeigt und die Option Passwort vergessen angeboten
			\item Die Email-adresse ist noch nicht registriert, es wird eine Fehlermedldung angezeigt, und die Registrierung angeboten
		\end{itemize} \\
		\hline Benuzte Use Cases &  \\
		\hline \multicolumn{2}{|c|}{Weitere Inforamtionen} \\
		\hline Spezielle Anforderungen &  \\
		\hline Annahmen &  \\
		\hline
		\end{tabular}
			 
	\subsubsection{Registrieren}
		\begin{tabular}{|l|p{.5\linewidth}|}
		\hline Use Case Nummer & 1.1.2 \\ 
		\hline Use Case Name & Registrieren \\ 
		\hline Initiirender Akteur & Benutzer \\
		\hline Weitere Akteure &  \\
		\hline Kurzbeschreibung & Ein neuer Benutzer kann sich mit seiner Email-Adresse anmelden \\
		\hline Vorbedingung & Benutzer hat auf Registrieren geklickt \\
		\hline Nachbedingung & Benutzer ist eingeloggt \\
		\hline \multicolumn{2}{|c|}{Funktionalitaet des UseCases}\\
		\hline Ablauf & \begin{itemize}
			\item Benutzer gibt Email-Adresse und Passwort ein
			\item Benutzer w\"ahlt fakult\"at
			\item Benutzer klickt "jetzt mitmachen"
		\end{itemize} \\
		\hline Alternativen &  \\
		\hline Ausnahmen & \begin{itemize}
			\item Email-adresse ist schon vergeben, es wird eine Fehlermeldung angezeigt
			\item das Passwort ist zu kurz, es wird eine Fehlermeldung angezeit
			\item Keine Fakult\"at gewahlt, es wir eine Fehlermeldung angezeigt
		\end{itemize} \\
		\hline Benuzte Use Cases &  \\
		\hline \multicolumn{2}{|c|}{Weitere Inforamtionen} \\
		\hline Spezielle Anforderungen &  \\
		\hline Annahmen &  \\
		\hline
		\end{tabular} 
		
	\subsubsection{Passwort vergessen}
		\begin{tabular}{|l|p{.5\linewidth}|}
		\hline Use Case Nummer & 1.1.3 \\ 
		\hline Use Case Name & Passwort vergessen \\ 
		\hline Initiirender Akteur & Benutzer \\
		\hline Weitere Akteure &  \\
		\hline Kurzbeschreibung & Ein Benutzer hat die M\"oglichkeit sein Passwort neu anzufordern \\
		\hline Vorbedingung & Benutzer ist nicht eingeloggt \\
		\hline Nachbedingung & Benutzer hat eine Email bekommen \\
		\hline \multicolumn{2}{|c|}{Funktionalitaet des UseCases}\\
		\hline Ablauf & \begin{itemize}
			\item Benutzer gibt seine Email-Adresse an
			\item Benutzer klickt auf "jetzt Passwort anfordern"
		\end{itemize} \\
		\hline Alternativen &  \\
		\hline Ausnahmen & Email-Adresse ist nicht registriert, es wird eine Fehlermeldung angezeigt \\
		\hline Benuzte Use Cases &  \\
		\hline \multicolumn{2}{|c|}{Weitere Inforamtionen} \\
		\hline Spezielle Anforderungen &  \\
		\hline Annahmen &  \\
		\hline
		\end{tabular}
	\subsubsection{Fehlermeldung}
		\begin{tabular}{|l|p{.5\linewidth}|}
		\hline Use Case Nummer & 1.1.4 \\ 
		\hline Use Case Name & Fehlermeldung \\ 
		\hline Initiirender Akteur & Benutzer \\
		\hline Weitere Akteure &  \\
		\hline Kurzbeschreibung & Es wird dem Benutzer eine Fehlermeldung angezeigt und beleibt auf der aktuellen Seite \\
		\hline Vorbedingung &  \\
		\hline Nachbedingung & Fehlermeldung wir angezeigt \\
		\hline \multicolumn{2}{|c|}{Funktionalitaet des UseCases}\\
		\hline Ablauf & Fehlermeldung wird in die Aktuelle Seite eingebettet \\
		\hline Alternativen &  \\
		\hline Ausnahmen &  \\
		\hline Benuzte Use Cases &  \\
		\hline \multicolumn{2}{|c|}{Weitere Inforamtionen} \\
		\hline Spezielle Anforderungen &  \\
		\hline Annahmen &  \\
		\hline
		\end{tabular} 
	 
% MUSTER 
	\subsubsection{title}
	\begin{tabular}{|l|p{.5\linewidth}|}
	\hline Use Case Nummer &  \\ 
	\hline Use Case Name &  \\ 
	\hline Initiirender Akteur &  \\
	\hline Weitere Akteure &  \\
	\hline Kurzbeschreibung &  \\
	\hline Vorbedingung &  \\
	\hline Nachbedingung &  \\
	\hline \multicolumn{2}{|c|}{Funktionalitaet des UseCases}\\
	\hline Ablauf &  \\
	\hline Alternativen &  \\
	\hline Ausnahmen &  \\
	\hline Benuzte Use Cases &  \\
	\hline \multicolumn{2}{|c|}{Weitere Inforamtionen} \\
	\hline Spezielle Anforderungen &  \\
	\hline Annahmen &  \\
	\hline
	\end{tabular} 

%------------------------------------------------
% Produktdaten
%------------------------------------------------
\section{Produktdaten}

%------------------------------------------------
% Produktleistung
%------------------------------------------------
\section{Produktleistung}

%------------------------------------------------
% Benutzeroberflache
%------------------------------------------------
\section{Benutzeroberfl\"ache}
\subsection{Webseite}
\subsection{Android App}
%------------------------------------------------
% Qualitaetsanforderung
%------------------------------------------------
\section{Qualit\"atsanforderung}

%------------------------------------------------
% Entwicklungsumgebung
%------------------------------------------------
\section{Entwicklungsumgebung}
\subsection{Software}
\begin{itemize}
	\item 
	\item Java 7
	\item Visual Paradigm Standard Edition 12
\end{itemize}
\subsection{Hardware}
\subsection{Orgware}
%------------------------------------------------
% Erganzungen
%------------------------------------------------
\section{Erg\"anzungen}

%------------------------------------------------
% Glossar
%------------------------------------------------
\section{Glossar}

Backend ist der Admin Bereich der Webseite

\end{document}