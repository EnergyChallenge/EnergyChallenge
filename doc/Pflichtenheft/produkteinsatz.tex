\section{Produkteinsatz}
\subsection{Anwendungsbereich}
Das hier beschriebene Softwaresystem wird im Umwelt- und Energiemanagementbereich eingesetzt. Es soll im universitären Betrieb die Nutzer  zum Energie sparen und umweltbewussten Leben animieren, motivieren und inspirieren. 
\subsection{Zielgruppen}
Leiter und Angestellte der Universität und deren Fakultäten sowie die Studentinnen und Studenten der Universität. 
Da die Bedienung im umgangssprachlichen Sinne intuitiv erfolgt und für die Benutzer die Interaktionsmöglichkeiten sowohl im Webbrowser als auch in einer App möglich sind, ist die Kenntnis von wenigen und einfachen Nutzungskonzepten von Software im Bereich des Social Media und Web 2.0 erforderlich. Ausgenommen hiervon ist der Administrator, an ihn werden leicht erhöhte, aber dennoch als geringe Anforderungen in Bezug auf Softwarekonfiguration, verteilte Anwendungen und Netzwerkverbindungen gestellt. 
Falls keine weiteren Sprachen integriert werden ist das Verständnis der deutschen Sprache erforderlich. 
\subsection{Betriebsbedingungen}
Durch die sorgfältige Planung des Systems ergeben sich folgende wichtige Kriterien: 
\begin{itemize}
\item Betriebsdauer: Dauerbetrieb 
\item Das System ist nicht wartungsfrei, da es regelmäßig benutzt wird 
\item Änderungen an dem Datenbestand werden sowohl von einem Administrator als auch von den Benutzern selbst durchgeführt 
\item Die Sicherung des Datenbestandes erfolgt automatisch und kann zusätzlich von einem Administrator manuell vorgenommen werden
\end{itemize}