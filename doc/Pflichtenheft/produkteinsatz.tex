\section{Produkteinsatz}
\subsection{Anwendungsbereich}
Das hier beschriebene Softwaresystem wird im Umwelt- und Energiemanagementbereich eingesetzt. Es soll im universit�ren Betrieb die Nutzer  zum Energie sparen und umweltbewussten Leben animieren, motivieren und inspirieren. \\
\subsection{Zielgruppen}
Leiter und Angestellte der Universit�t und deren Fakult�ten sowie die Studentinnen und Studenten der Universit�t. \\
Da die Bedienung im umgangssprachlichen Sinne intuitiv erfolgt und f�r die Benutzer die Interaktionsm�glichkeiten sowohl im Webbrowser als auch in einer App m�glich sind, ist die Kenntnis von wenigen und einfachen Nutzungskonzepten von Software im Bereich des Social Media und Web 2.0 erforderlich. Ausgenommen hiervon ist der Administrator, an ihn werden leicht erh�hte, aber dennoch als geringe Anforderungen in Bezug auf Softwarekonfiguration, verteilte Anwendungen und Netzwerkverbindungen gestellt. \\
Falls keine weiteren Sprachen integriert werden ist das Verst�ndnis der deutschen Sprache erforderlich. \\
\subsection{Betriebsbedingungen}
Durch die sorgf�ltige Planung des Systems ergeben sich folgende wichtige Kriterien: \\
\{itemize}
\item Betriebsdauer: Dauerbetrieb \\
\item Das System ist nicht wartungsfrei, da es regelm��ig benutzt wird \\
\item �nderungen an dem Datenbestand werden sowohl von einem Administrator als auch von den Benutzern selbst durchgef�hrt \\
\item Die Sicherung des Datenbestandes erfolgt automatisch und kann zus�tzlich von einem Administrator manuel vorgenommen werden\\
\end{itemize}